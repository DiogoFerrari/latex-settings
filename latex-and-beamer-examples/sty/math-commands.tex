% Created 2020-08-20 Thu 15:55
% Intended LaTeX compiler: pdflatex
% Template: Diogo Ferrari
\documentclass[a4paper]{article}
% === Packages =================================
\usepackage{./sty/basic-article}
\usepackage{./sty/math-commands}
\usepackage{./sty/math-commands-thm}
% === Document =================================
\author{diogo}
\date{\today}
\title{}
\begin{document}

\%\% \texttt{===============================================}
\%\% my-math-commands.sty
\%\% --------------------
\%\% Author: Diogo Ferrari
\%\%
\%\%
\%\% latex template for mathematical anotation
\%\% Last version: Dez/2015
\%=================================================


\ProvidesPackage{math-commands}
\RequirePackage{tikz}
\%=================================================
\%		Letters and math symbols
\%=================================================
\%\% -------------
\%\% Prob and Stat
\%\% -------------
\newcommand{\F}\{\mathcal{F}\}									   \%- funcy f
\newcommand{\Sa}\{\mathcal{S}\}                                      \%- sample space
\newcommand{\Ev}\{\mathcal{E}\}                                      \%- event
\newcommand{\U}\{\mathcal{U}\}									   \%- Universe
\newcommand{\M}\{\mathcal{M}\}									   \%- Model
\newcommand{\Hf}\{\mathcal{H}\}									   \%- func or set H
\newcommand{\Lf}\{\mathcal{L}\}									   \%- func or set L
\newcommand{\If}\{\mathcal{I}\}									   \%- func or set I
\newcommand{\Zf}\{\mathcal{Z}\}									   \%- func or set Z
\newcommand{\Ps}\{\mathcal{P}\}									   \%- set of probability distributions
\newcommand{\E}\{\mathbb{E}\}                                        \%- expectation
\newcommand{\Var}\{\mathbb{V}\text{ar}\}                             \%- variance
\newcommand{\Cov}\{\mathbb{C}\text{ov}\}                             \%- covariance
\newcommand{\Corr}\{\mathbb{C}\text{orr}\}		                   \%- correlation
\newcommand{\supp}\{\text{supp}\}                                    \%- support 
\newcommand{\given}{\mid}                                          \%- given/conditional
\newcommand{\bgiven}{\middle\vert}                                 \%- big given/conditional
\newcommand{\I}\{\mathbb{I}\}                                        \%- Indicator function
\newcommand{\rmatr}[1]\{\mathbf{#1}\}                                \%- random matrix (bold upper case)
\%\% \DeclareMathOperator*{\plim}{plim}                                 \%- prob limit
\newcommand{\plim}\{\operatorname*{plim}\}
\newcommand\ind\{\protect\mathpalette{\protect\independenT}{\perp}\} \%- stoch independence
\def\independenT\#1\#2\{\mathrel\{\rlap{$#1#2$}\mkern2mu\{\#1\#2\}\}\}       \%- stoch independence
\newcommand{\iid}\{\overset{iid}{\sim}\}                             \%- iid
\newcommand{\dist}{\sim}                        \%- distributed as
\newcommand{\distas}[1]\{\overset{#1}{\sim}\}                        \%- distributed as
\newcommand{\convdist}\{\overset{d}{\rightarrow}\}                   \%- conv in distribution
\newcommand{\convprob}\{\overset{p}{\rightarrow}\}                   \%- conv in probability
\newcommand{\convas}\{\overset{a.s.}{\rightarrow}\}                  \%- conv almost sure
\%\% Distributions
\newcommand{\supp}\{\text{supp}\}                                    \%- support of a distribution
\newcommand{\No}\{\mathcal{N}\}                                      \%- Normal Distribution
\newcommand{\DP}\{\mathcal{D}\mathcal{P}\}                           \%- Diriclhet Process
\%\% -----------------
\%\% Sets/Set notation
\%\% -----------------
\%\% \DeclarePairedDelimiterX*\set[1]\lbrace\rbrace\{
\%\%                          \def\given{\;\delimsize\vert\;}\#1\}        \%- Set delimiters
\newcommand{\set}[1]{\Big\lbrace #1 \Big\rbrace}\%     \{\}
\newcommand{\setrl}[1]{\left\lbrace #1 \right\rbrace}\%     \{\}
\newcommand{\N}\{\mathbb{N}\}                                        \%- Set of Natural Numbers
\newcommand{\Z}\{\mathbb{Z}\}                                        \%- Set of Integers
\newcommand{\Q}\{\mathbb{Q}\}                                        \%- Set of rationals
\newcommand{\R}\{\mathbb{R}\}                                        \%- Set of reals
\newcommand{\Co}\{\mathbb{C}\}                                        \%- Set of complex numbers
\newcommand{\A}\{\mathcal{A}\}									   \%- Set  A
\newcommand{\B}\{\mathcal{B}\}									   \%- Set  B
\newcommand{\C}\{\mathcal{C}\}									   \%- Set  C
\newcommand{\D}\{\mathcal{D}\}									   \%- Set  D
\newcommand{\G}\{\mathcal{G}\}									   \%- Set  G
\newcommand{\X}\{\mathcal{X}\}									   \%- Set  X
\newcommand{\Zs}\{\mathcal{Z}\}									   \%- Set  X
\newcommand{\Rs}\{\mathcal{R}\}									   \%- Set  R
\newcommand{\Y}\{\mathcal{Y}\}									   \%- Set  Y
\newcommand{\card}[1]{\left|#1\right|}                             \%- cardinality 1
\newcommand{\cardi}\{\text{card}\}                             \%- cardinality 2
\newcommand{\aff}\{\text{aff}\}                                        \%- Affine hull
\newcommand{\conv}\{\text{conv}\}                                        \%- Convex hull
\newcommand{\coni}\{\text{coni}\}                                        \%- conic hull
\newcommand{\span}\{\text{span}\}                                        \%- span
\newcommand{\union}{\bigcup}                                        \%- disjoint union
\newcommand{\dunion}{\bigsqcup}                                        \%- disjoint union

\%\% -----
\%\% Logic
\%\% -----
\newcommand{\AND}{\wedge}                                          \%- and
\newcommand{\OR}{\vee}                                             \%- or
\newcommand{\notimplies}{\centernot\implies}                       \%- does not imply
\%\% --------------
\%\% Linear Algebra
\%\% --------------
\newcommand{\inner}[1]{\langle#1\rangle}                           \%- inner product
\%\% --------
\%\% Calculus
\%\% --------
\newcommand{\p}{\partial}                                          \%- partial derivatives
\%\% \DeclareMathOperator*{\limsup}{lim sup}                            \%- lim sup    
\%\% \DeclareMathOperator*{\liminf}{lim inf}                            \%- lim inf
\%\% \DeclareMathOperator*{\argmax}{argmax}                             \%- argmax
\%\% \DeclareMathOperator*{\argmin}{argmin}                             \%- argmin
\newcommand{\limsup}\{\operatorname*{limsup}\}
\newcommand{\liminf}\{\operatorname*{liminf}\}
\newcommand{\argmin}\{\operatorname*{argmin}\}
\newcommand{\argmax}\{\operatorname*{argmax}\}
\newcommand{\grad}{\nabla}                                         \%- gradient
\newcommand{\dom}\{\text{dom}\}                                      \%- domain of f
\newcommand{\img}\{\text{img}\}                                      \%- image of f
\newcommand\at[2]\{\left.\#1\right|\_\{\#2\}\}                            \%- derivative at
\newcommand{\Lagr}\{\mathcal{L}\}                                    \%- Lagrangian
\newcommand{\e}{\varepsilon}                                       \%- epsilon
\%\% ------
\%\% Others
\%\% ------
\newcommand{\vect}[1]\{\boldsymbol{#1}\}                             \%- vector bold font
\newcommand{\vectm}[1]\{\mathbf{#1}\}                             \%- vector bold font for matrices
\newcommand{\abs}[1]{\left|#1\right|}                             \%- module
\newcommand{\erf}[1]\{\text{erf}\left(\#1\right)\}                    \%- error function
\newcommand{\defined}{\triangleq}                                      \%- definition
\newcommand{\definedas}{\triangleq}                                      \%- definition
\newcommand{\norm}[1]{\lVert#1\rVert}                             \%- norm
\usepackage{mathtools}
\%\% \DeclarePairedDelimiter\ceil{\lceil}{\rceil}                    \%- ceiling func
\%\% \DeclarePairedDelimiter\floor{\lfloor}{\rfloor}                 \%- floor func
\%\% condition environment, for description of vars (works similar to align)
\newenvironment{conditions}
\{\par\vspace{\abovedisplayskip}\noindent\begin{tabular}\{>\{\(}l<{\)\} @\{\({}={}\)\} l\}\}
\{\end{tabular}\par\vspace{\belowdisplayskip}\}







\%\% \texttt{==============================================}
\%\% For graphs
\%\% \texttt{==============================================}
\% tikzlibrary.code.tex
\% Modified from \url{https://github.com/jluttine/tikz-bayesnet}
\% 
\% Copyright 2010-2011 by Laura Dietz
\% Copyright 2012 by Jaakko Luttinen
\%
\% This file may be distributed and/or modified
\%
\% 1. under the \LaTeX{} Project Public License and/or
\% 2. under the GNU General Public License.
\%
\% See the files LICENSE\_LPPL and LICENSE\_GPL for more details.

\% Load other libraries
\usetikzlibrary{shapes}
\usetikzlibrary{fit}
\usetikzlibrary{chains}
\usetikzlibrary{arrows}
\usetikzlibrary{petri}
\%======
\% Nodes
\%======
\usetikzlibrary{shadows.blur}
\usetikzlibrary{shapes.symbols}
\newcommand{\DAGnodedistance}{30pt}
\newcommand{\DAGinnersep}{5pt}
\newcommand{\DAGminsize}{20pt}
\newcommand{\DAGfont}\{\fontsize{10}{10}\selectfont\}
\newcommand{\DAGcolorfont}{black}
\newcommand{\DAGcolorborder}{black}
\newcommand{\DAGcolorfill}{white}
\newcommand{\DAGlinewidth}{.7pt}
\tikzstyle{basic} = [
    shape         =circle, 
    draw          =\DAGcolorborder,
    line width    =\DAGlinewidth,
    minimum size  =\DAGminsize,
    inner sep     =\DAGinnersep,
    font          =\DAGfont,
    text          =\DAGcolorfont,
    fill          =\DAGcolorfill,
    node distance =\DAGnodedistance,                  \% for relative positions
    blur shadow=\{shadow blur steps=5\}
    ]
\tikzstyle{obs}            = [basic]                        \% Latent node
\tikzstyle{obs2}           = [basic, fill=gray!25]          \% Observed node
\tikzstyle{latent}         = [basic, fill=gray!25]          \% Observed node
\%\% \tikzstyle{factor}         = [basic, fill=black, text=white]\% Factor node
\tikzstyle{factor}         = [rectangle, fill=black,minimum size=5pt, inner sep=0pt, node distance=0.4]
\tikzstyle{factor caption} = [caption] \%
\tikzstyle{const}          = [basic, rectangle,]            \% Constant node
\tikzstyle{det}            = [basic, inner sep     =1pt, diamond]               \% Deterministic node
\tikzstyle{dist}           = [rectangle, draw, fill=black,minimum size=10pt, inner sep=0pt, node distance=0.4]
\tikzstyle{operation}      = [basic, inner sep     =1pt, diamond]               \% Deterministic node


\% Plate node
\% ---------- 
\tikzset\{
  plate/.style=\{
    draw = black,
    shape=rectangle,
    rounded corners=0.5ex,
    thick,
    minimum width=3.1cm,
    text width=3.1cm,
    align=right,
    inner sep=10pt,
    inner ysep=10pt,
  \}
\}
\newcommand{\DAGplateinnersep}{15pt}
\newcommand{\DAGplatecolorborder}{black}
\tikzstyle{plate caption} = [
  caption,
  node distance=0,
  inner sep=0pt,
  below left=0pt and 0pt of \#1.south east] \%
\tikzstyle{plate} = [
  draw=black,
  text width=3.1cm,
  shape=rectangle,
  solid,           \% dashed, dotted
  rounded corners,
  fit=\#1,
  color         = \DAGplatecolorborder,
  inner sep     = \DAGplateinnersep,
  xshift=0cm,   \% displacement to x direcation
  yshift=0cm,   \% displacement to y direcation
  node distance=5pt, 
]
\tikzstyle{wrap}  = [inner sep=0pt, fit=\#1]\% Invisible wrapper node
\tikzstyle{gate}  = [draw, rectangle, dashed, fit=\#1]\% Gate

\% Caption node
\% ------------ 
\tikzstyle{caption} = [font=\footnotesize, node distance=0] \%
\tikzstyle{every label} += [caption] \%

\tikzset\{>=\{triangle 45\}\}

\%\pgfdeclarelayer{b}
\%\pgfdeclarelayer{f}
\%\pgfsetlayers{b,main,f}

\% \factoredge [options] \{inputs\} \{factors\} \{outputs\}
\newcommand{\factoredge}[4][]\{ \%
  \% Connect all nodes \#2 to all nodes \#4 via all factors \#3.
  \foreach \f in \{\#3\} \{ \%
    \foreach \x in \{\#2\} \{ \%
      \path (\x) edge[-,\#1] (\f) ; \%
      \%\draw[-,#1] (\x) edge[-] (\f) ; \%
    \} ;
    \foreach \y in \{\#4\} \{ \%
      \path (\f) edge[->,\#1] (\y) ; \%
      \%\draw[->,#1] (\f) -- (\y) ; \%
    \} ;
  \} ;
\}

\% \edge [options] \{inputs\} \{outputs\}
\newcommand{\edge}[3][]\{ \%
  \% Connect all nodes \#2 to all nodes \#3.
  \foreach \x in \{\#2\} \{ \%
    \foreach \y in \{\#3\} \{ \%
      \path (\x) edge [->, >=stealth',shorten >=1pt, \#1] (\y) ;\%
      \%\draw[->,#1] (\x) -- (\y) ;\%
    \} ;
  \} ;
\}

\% \edge [options] \{inputs\} \{outputs\}
\newcommand{\edgelat}[3][]\{ \%
  \% Connect all nodes \#2 to all nodes \#3.
  \foreach \x in \{\#2\} \{ \%
    \foreach \y in \{\#3\} \{ \%
      \path (\x) edge [->, dashed,>=stealth',shorten >=1pt,  \#1] (\y) ;\%
      \%\draw[->,#1] (\x) -- (\y) ;\%
    \} ;
  \} ;
\}

\% \edge [options] \{inputs\} \{outputs\}
\newcommand{\edgelatbidir}[3][]\{ \%
  \% Connect all nodes \#2 to all nodes \#3.
  \foreach \x in \{\#2\} \{ \%
    \foreach \y in \{\#3\} \{ \%
      \path (\x) edge [<->, >=stealth',shorten >=1pt, dashed, bend left=50, \#1] (\y) ;\%
      \%\draw[->,#1] (\x) -- (\y) ;\%
    \} ;
  \} ;
\}



\% \factor [options] \{name\} \{caption\} \{inputs\} \{outputs\}
\newcommand{\factor}[5][]\{ \%
  \% Draw the factor node. Use alias to allow empty names.
  \node[factor, label=\{[name=\#2-caption]\#3\}, name=\#2, \#1,
  alias=\#2-alias] \{\} ; \%
  \% Connect all inputs to outputs via this factor
  \factoredge \{\#4\} \{\#2-alias\} \{\#5\} ; \%
\}

\% \plate [options] \{name\} \{fitlist\} \{caption\}
\newcommand{\plate}[4][]\{ \%
  \node[wrap=#3] (\#2-wrap) \{\}; \%
  \node[plate caption=#2-wrap] (\#2-caption) \{\#4\}; \%
  \node[plate=(#2-wrap)(#2-caption), #1] (\#2) \{\}; \%
\}

\% \gate [options] \{name\} \{fitlist\} \{inputs\}
\newcommand{\gate}[4][]\{ \%
  \node[gate=#3, name=#2, #1, alias=#2-alias] \{\}; \%
  \foreach \x in \{\#4\} \{ \%
    \draw [-*,thick] (\x) -- (\#2-alias); \%
  \} ;\%
\}

\% \vgate \{name\} \{fitlist-left\} \{caption-left\} \{fitlist-right\}
\% \{caption-right\} \{inputs\}
\newcommand{\vgate}[6]\{ \%
  \% Wrap the left and right parts
  \node[wrap=#2] (\#1-left) \{\}; \%
  \node[wrap=#4] (\#1-right) \{\}; \%
  \% Draw the gate
  \node[gate=(#1-left)(#1-right)] (\#1) \{\}; \%
  \% Add captions
  \node[caption, below left=of #1.north ] (\#1-left-caption)
  \{\#3\}; \%
  \node[caption, below right=of #1.north ] (\#1-right-caption)
  \{\#5\}; \%
  \% Draw middle separation
  \draw [-, dashed] (\#1.north) -- (\#1.south); \%
  \% Draw inputs
  \foreach \x in \{\#6\} \{ \%
    \draw [-*,thick] (\x) -- (\#1); \%
  \} ;\%
\}

\% \hgate \{name\} \{fitlist-top\} \{caption-top\} \{fitlist-bottom\}
\% \{caption-bottom\} \{inputs\}
\newcommand{\hgate}[6]\{ \%
  \% Wrap the left and right parts
  \node[wrap=#2] (\#1-top) \{\}; \%
  \node[wrap=#4] (\#1-bottom) \{\}; \%
  \% Draw the gate
  \node[gate=(#1-top)(#1-bottom)] (\#1) \{\}; \%
  \% Add captions
  \node[caption, above right=of #1.west ] (\#1-top-caption)
  \{\#3\}; \%
  \node[caption, below right=of #1.west ] (\#1-bottom-caption)
  \{\#5\}; \%
  \% Draw middle separation
  \draw [-, dashed] (\#1.west) -- (\#1.east); \%
  \% Draw inputs
  \foreach \x in \{\#6\} \{ \%
    \draw [-*,thick] (\x) -- (\#1); \%
  \} ;\%
\}

\% End graphs
\%==================================================


\% For DAG concepts in using math notation (ancestor, descentant, etc)
\% ---------------------------------------------------------- 
\newcommand\des[1]\{\underaccent{\bullet}{#1}\}
\newcommand\ndes[1]\{\underaccent\{\slashed{\bullet}\}\{\#1\}\}
\newcommand\child[1]\{\underaccent{\widecheck}{#1}\}
\newcommand\nanc[1]\{\accentset\{\bm\slashed{\circ}\}\{\#1\}\}
\newcommand\anc[1]\{\accentset{\circ}{#1}\}
\%\% \newcommand\anc[1]\{\accentset{\triangledown}{#1}\}
\%\% \newcommand\nanc[1]\{\accentset\{\slashed{\triangledown}\}\{\#1\}\}
\newcommand\pa[1]\{\widecheck{#1}\}
\newcommand\doo[1]\{\widetilde{#1}\}
\newcommand\obar[1]\{\widebar{#1}\}
\newcommand{\ubar}[1]\{\underaccent{\widebar}{#1}\}




\%==================================================
\% emphasis box around equation environment
\%==================================================
\usepackage[overload]{empheq}
\newcommand*{\widebox}[2][0.5em]\{\fbox\{\hspace{#1}\$\displaystyle \#2\$\hspace{#1}\}\}
\usepackage{xpatch}\%
\makeatletter
\xpatchcmd{\@Aboxed}\{\%
\boxed \{\#1\#2\}\}
\{\%
\color{HotPink3}\boxed \{\color{textcolor}\#1\#2\}\}
\{\}\{\}
\makeatother
\end{document}
